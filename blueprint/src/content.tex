% In this file you should put the actual content of the blueprint.
% It will be used both by the web and the print version.
% It should *not* include the \begin{document}
%
% If you want to split the blueprint content into several files then
% the current file can be a simple sequence of \input. Otherwise It
% can start with a \section or \chapter for instance.

\section{Basic definitions for sphere packings}
\subsection{Sphere packings}
The sphere packing constant measures which portion of $d$-dimensional Euclidean space can be covered by nonoverlapping unit balls.

\subsection{Dummy Lemmas}

Here are some dummy lemmas to test whether blueprint and its Lean interface actually work.

\begin{lemma}\label{foo1}\lean{SpherePacking.DummyLemmas.foo1}\leanok
  For all natural numbers $n$, we have that $n = n$.
\end{lemma}
\begin{proof}
  \leanok
  Proof by reflexivity.
\end{proof}

\begin{lemma}\label{foo2}\lean{SpherePacking.DummyLemmas.foo2}\uses{foo1}\leanok
  For all integers $n$, we have that $n = n$.
\end{lemma}
\begin{proof}
  This is super hard to prove so we will \verb|sorry| it for now.
\end{proof}

\begin{lemma}\label{foo3}\lean{SpherePacking.DummyLemmas.foo3}\leanok
  $1 + 1 = 2$.
\end{lemma}
\begin{proof}
  \leanok
  Proof by computation.
\end{proof}

Let's turn things up a notch.

\begin{lemma}\label{foo4}\lean{SpherePacking.DummyLemmas.foo4}\leanok  % Marking \leanok even though it hasn't been formalised
  There exists an isomorphism
  \[
    \R[X] / (X^2 + 1) \cong \C
  \]
\end{lemma}
\begin{proof}
  \leanok
  Proof by next level algebra awesomeness.
\end{proof}
