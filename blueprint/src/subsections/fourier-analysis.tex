In this section, we recall a few definitions from Fourier analysis.

\begin{definition}\label{def-Fourier-Transform} % \lean{def-Fourier-Transform} % Specialise Mathlib.VectorFourier.fourierIntegral
  The Fourier transform of an $L^1$-function $f:\R^d\to\C$ is defined as

  \[
    \mathcal{F}(f)(y) = \widehat{f}(y) \coloneqq \int_{\R^d} f(x)e^{-2\pi i \langle x, y \rangle} \,\mathrm{d}x, \quad y \in \R^d
  \]

  where $\langle x, y \rangle = \frac12\|x\|^2 + \frac12\|y\|^2 - \frac12\|x - y\|^2$ is the standard scalar product in $\R^d$.
\end{definition}

\begin{definition}\label{def-Schwartz-Function}
A $C^\infty$~function $f:\R^d\to\C$ is called a \emph{Schwartz function} if it goes to zero as $\|x\|\to\infty$ faster then any inverse power of $\|x\|$, and the same holds for all partial derivatives of $f$.
\end{definition}

\begin{definition}\label{def-Schwartz-Space}\uses{def-Schwartz-Function}
    The set of all Schwartz functions is called a \emph{Schwartz space}.
\end{definition}

\begin{lemma}\label{lemma-Fourier-transform-is-automorphism}\uses{def-Fourier-Transform, def-Schwartz-Space}
  The Fourier transform is an automorphism of the space of Schwartz functions.
\end{lemma}
\begin{proof}
  \todo{Fill in proof.}
\end{proof}

\begin{lemma}\label{lemma-Gaussian-Fourier}\uses{def-Fourier-Transform}
  \begin{equation}
    \mathcal{F}(e^{\pi i \|x\|^2 z})(y) = z^{-4}\,e^{\pi i \|y\|^2 \,(\frac{-1}{z}) }.
  \end{equation}
\end{lemma}
\begin{proof}
  \todo{Fill in proof.}
\end{proof}

\begin{theorem}[(Poisson summation formula)]\label{thm-Poisson-summation-formula}\uses{def-Fourier-Transform}
  $$\sum_{\ell\in\Lambda}f(\ell)=\frac{1}{\mathrm{Vol}(\R^d/\Lambda)}\sum_{m\in\Lambda^*}\widehat{f}(m).$$
\end{theorem}
\begin{proof}
  \todo{Fill in proof.}
\end{proof}
