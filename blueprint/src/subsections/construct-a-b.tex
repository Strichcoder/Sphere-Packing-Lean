% Gareth: I haven't sorted the content of this file yet. We definitely should
% break this file down further, but also we should consider whether to
% formalise this proof or an alternative one like Seewoo's or Dan's.

In this section we construct two radial Schwartz functions $a,b:\R^8\to i\R$ such that
\begin{align}\mathcal{F}(a)&=a\label{eqn:a-fourier}\\
    \mathcal{F}(b)&=-b\label{eqn:b-fourier}
\end{align}
which double zeroes at all $\Lambda_8$-vectors of length greater than $\sqrt{2}$. Recall that each vector of $\Lambda_8$ has length $\sqrt{2n}$ for some $n\in\N_{\geq 0}$. We define $a$ and $b$ so that their values are purely imaginary because this simplifies some of our computations. We will show in Section \ref{sec: g} that an appropriate linear combination of functions $a$ and $b$ satisfies conditions \eqref{eqn:g1}--\eqref{eqn:g3}.

First, we will define function $a$. To this end we consider the following functions:
\begin{definition}\label{def:phi4-phi2-phi0}
\uses{def:Ek,def:E2}
\begin{align}
    \phi_{-4} &:= \frac{E_4^2}{\Delta} \label{eqn: def phi4} \\
    \phi_{-2} &:= \frac{E_4(E_2 E_4 - E_6)}{\Delta} \label{eqn: def phi2} \\
    \phi_{0} &:= \frac{(E_2 E_4 - E_6)^2}{\Delta} \label{eqn: def phi0}
\end{align}
\end{definition}

% \begin{lemma}\label{lemma: phi fourier4 phi fourier2 phi fourier0}
% \uses{def:phi4-phi2-phi0}
%     These functions have the Fourier expansions
% \begin{align}
%     \phi_{-4}(z)\,=\,&q^{-1} + 504 + 73764\, q + 2695040\, q^2 + 54755730\, q^3 + O(q^4)\label{eqn: phi fourier4}\\
%     \phi_{-2}(z)\,=\,&720 + 203040\, q + 9417600\, q^2 + 223473600\, q^3 + 3566782080\, q^4+O(q^5)\label{eqn: phi fourier2}\\
%     \phi_{0}(z)\,=\,&518400\, q + 31104000\, q^2 + 870912000\, q^3 + 15697152000\, q^4+O(q^5)\label{eqn: phi fourier0}
% \end{align}
% where $q=e^{2\pi i z}$.
% \end{lemma}

The function $\phi_0(z)$ is not modular; however, it satisfies the following transformation rules:

\begin{lemma}\label{lemma:phi0-transform}\uses{def:phi4-phi2-phi0, lemma:Ek-is-modular-form, lemma:E2-transform, lemma:disc-cuspform}
We have
\begin{align}
    \phi_0(z + 1) &= \phi_0(z) \label{eqn:phi0-trans-T} \\
    \phi_0\left(-\frac{1}{z}\right) &= \phi_0(z)-\frac{12i}{\pi}\,\frac{1}{z}\,\phi_{-2}(z)-\frac{36}{\pi^2}\,\frac{1}{z^2}\,\phi_{-4}(z).\label{eqn:phi0-trans-S}
\end{align}
\end{lemma}
\begin{proof}
\eqref{eqn:phi0-trans-T} easily follows from periodicity of Eisenstein series and $\Delta(z)$.
For \eqref{eqn:phi0-trans-S},
\begin{align}
    \phi_{0}\left(-\frac{1}{z}\right) &= \frac{(E_2(-1/z) E_4(-1/z) - E_6(-1/z))^{2}}{\Delta(-1/z)} \\
    &= \frac{((z^2 E_2(z) - 6iz / \pi) \cdot z^4 E_4(z) - z^6 E_6(z))^{2}}{z^{12} \Delta(z)} \\
    &= \frac{\left(E_2(z) E_4(z) - E_6(z) - \frac{6i}{\pi z} E_4(z)\right)^2}{\Delta(z)} \\
    &= \frac{(E_2(z) E_4(z) - E_6(z))^{2} - \frac{12i}{\pi z}(E_2(z) E_4(z) - E_6(z)) E_4(z) - \frac{36}{\pi^2 z^2} E_4(z)^{2}}{\Delta(z)} \\
    &= \phi_0(z) - \frac{12 i}{\pi z} \phi_{-2}(z) - \frac{36}{\pi^2 z^2} \phi_{-4}(z).
\end{align}
\end{proof}

\begin{definition}\label{def:a-definition}
\uses{def:phi4-phi2-phi0}
For $x\in\R^8$ we define
\begin{align}\label{eqn:a-definition}
    a(x):=&\int\limits_{-1}^i\phi_0\Big(\frac{-1}{z+1}\Big)\,(z+1)^2\,e^{\pi i \|x\|^2 z}\,dz
    +\int\limits_{1}^i\phi_0\Big(\frac{-1}{z-1}\Big)\,(z-1)^2\,e^{\pi i \|x\|^2 z}\,dz\\
    -&2\int\limits_{0}^i\phi_0\Big(\frac{-1}{z}\Big)\,z^2\,e^{\pi i \|x\|^2 z}\,dz
    +2\int\limits_{i}^{i\infty}\phi_0(z)\,e^{\pi i \|x\|^2 z}\,dz.\nonumber
\end{align}
\end{definition}

We observe that the contour integrals in \eqref{eqn:a-definition} converge absolutely and uniformly for $x\in\R^8$.
Indeed, $\phi_0(z)=O(e^{-2\pi i z})$ as $\Im(z)\to \infty$. Therefore, $a(x)$ is well defined. Now we prove that $a$ satisfies condition \eqref{eqn:a-fourier}.
The following lemma will be used to prove Schwartzness of $a$ and $b$.

\begin{lemma}\label{lemma:mod-div-disc-bound}\uses{lemma:disc-prodformula}
Let $f(z)$ be a holomorphic function with a Fourier expansion
\begin{equation}
    f(z) = \sum_{n \ge n_0} c_f(n) e^{\pi i n z}
\end{equation}
with $c_f(n_0) \ne 0$.
Assume that $c_f(n)$ has a polynomial growth, i.e. $|c_f(n)| = O(n^k)$ for some $k \in \N$.
Then there exists a constant $C_f > 0$ such that
\begin{equation}
    \left|\frac{f(z)}{\Delta(z)}\right| \le C_f e^{-\pi (n_0 - 2) \Im z}
\end{equation}
for all $z$ with $\Im z > 1/2$.
\end{lemma}
\todo{
Note that the assumption on the polynomial growth holds when $f$ is a holomorphic modular form, where the proof can be found in \cite[p. 94]{Serre73} for the case of level 1 modular forms.
But we just add this for simplicity, and we can prove it for ``specific'' $f$ such as Eisenstein series, theta functions, and their combinations.
}
\begin{proof}
By the product formula \eqref{eqn:disc-prodformula},
\begin{align}
    \left|\frac{f(z)}{\Delta(z)}\right| &= \left|\frac{\sum_{n \ge n_0} c_f(n) e^{\pi i n z}}{e^{2 \pi i z}\prod_{n \ge 1} (1 - e^{2\pi i n z})^{24}}\right| \\
    &= |e^{\pi i (n_0 - 2)z}| \cdot \frac{|\sum_{n \ge n_0} c_f(n) e^{\pi i (n - n_0) z}|}{\prod_{n \ge 1} |1 - e^{2\pi i n z}|^{24}} \\
    &\le e^{-\pi (n_0 - 2) \Im z} \cdot \frac{\sum_{n \ge n_0} |c_f(n)| e^{-\pi (n - n_0) \Im z}}{\prod_{n \ge 1} (1 - e^{- 2\pi n \Im z})^{24}} \\
    &\le e^{-\pi (n_0 - 2) \Im z} \cdot \frac{\sum_{n \ge n_0} |c_f(n)| e^{-\pi (n - n_0) / 2}}{\prod_{n \ge 1} (1 - e^{-\pi n})^{24}} \\
    &= C_f \cdot e^{-\pi (n_0 - 2) \Im z}
\end{align}
where
\begin{equation}
    C_f = \frac{\sum_{n \ge n_0} |c_f(n)| e^{-\pi (n - n_0) / 2}}{\prod_{n \ge 1} (1 - e^{-\pi n})^{24}}.
\end{equation}
Note that the summation in the numerator converges absolutely because of polynomial growth.
The denominator also converges, which is simiply $e^{\pi} \cdot \Delta(i/2)$.
\end{proof}

As corollaries, we have the following bound for $\phi_0$, $\phi_{-2}$, and $\phi_{-4}$.

\begin{corollary}\label{cor:phi0-bound}\uses{thm:ramanujan-formula, lemma:mod-div-disc-bound, lemma:Ek-Fourier}
There exists a constant $C_0 > 0$ such that
\begin{equation}\label{eqn:phi0-bound}
|\phi_0(z)| \le C_0 e^{-2 \pi \Im z}
\end{equation}
for all $z$ with $\Im z > 1/2$.
\end{corollary}
\begin{proof}
By Ramanujan's formula, $E_2 E_4 - E_6 = 3E_4' = 720 \sum_{n \ge 1} n \sigma_3(n) e^{2 \pi i n z}$ and
\begin{equation}
    (E_2(z) E_4(z) - E_6(z))^{2} = 720^{2} e^{4 \pi i z} + O(e^{5 \pi i z}). \notag
\end{equation}
Then the result follows from Lemma~\ref{lemma:mod-div-disc-bound} with $f(z) = (E_2 E_4 - E_6)^2$ and $n_0 = 4$.
\end{proof}

\begin{corollary}\label{cor:phi2-bound}\uses{def:phi4-phi2-phi0, lemma:mod-div-disc-bound, lemma:Ek-Fourier}
There exists a constant $C_{-2} > 0$ such that
\begin{equation}\label{eqn:phi2-bound}
    |\phi_{-2}(z)| \le C_{-2}
\end{equation}
for all $z$ with $\Im z > 1/2$.
\end{corollary}

\begin{corollary}\label{cor:phi4-bound}\uses{def:phi4-phi2-phi0, lemma:mod-div-disc-bound, lemma:Ek-Fourier}
There exists a constant $C_{-4} > 0$ such that
\begin{equation}\label{eqn:phi4-bound}
    |\phi_{-4}(z)| \le C_{-4} e^{2 \pi \Im z}
\end{equation}
for all $z$ with $\Im z > 1/2$.
\end{corollary}

Note that we can take the constants $C_0$, $C_{-2}$, and $C_{-4}$ as
\begin{align}
    C_0 &= 9 \cdot 240^2 \cdot e^{\pi} \cdot \frac{E_4'(i/2)^{2}}{\Delta(i/2)} \notag \\
    C_{-2} &= 3 \cdot \frac{E_4(i/2) E_4'(i/2)}{\Delta(i/2)} \notag \\
    C_{-4} &= e^{-\pi} \cdot \frac{E_4(i/2)^{2}}{\Delta(i/2)}. \notag
\end{align}

\begin{proposition}\label{prop:a-schwartz}
\uses{def:a-definition, cor:phi0-bound}
$a(x)$ is a Schwartz function.
\end{proposition}
\begin{proof}
We estimate the first summand in the right-hand side of \eqref{eqn:a-definition}.
By \eqref{eqn:phi0-bound}, we have
\begin{align}
    &\left|\int\limits_{-1}^{i}\phi_0\Big(\frac{-1}{z+1}\Big)\,(z+1)^2\,e^{\pi i r^2 z}\,dz\right|=\left|\int\limits_{i\infty}^{-1/(i+1)}\phi_0(z)\,z^{-4}\,e^{\pi i r^2 (-1/z-1)}\,dz\right|\leq \notag\\
    &C_1\int\limits_{1/2}^{\infty}e^{-2\pi t}\,e^{-\pi    r^2/t}\,dt\leq C_1\int\limits_{0}^{\infty}e^{-2\pi t}\,e^{-\pi    r^2/t}\,dt=C_2\,r\,K_1(2\sqrt{2}\,\pi\,r)\notag
\end{align}
where $C_1$ and $C_2$ are some positive constants and $K_\alpha(x)$ is the modified Bessel function of the second kind defined as in \cite[Section~9.6]{Abramowitz}. This estimate also holds for the second and third summand in \eqref{eqn:a-definition}.
For the last summand we have
\begin{equation}
\left|\int\limits_{i}^{i\infty}\phi_0(z)\,e^{\pi i r^2 z}\,dz\right|\leq C\,\int\limits_{1}^{\infty} e^{-2\pi t}\,e^{-\pi r^2 t}\,dt=C_3\frac{e^{\pi(r^2+2)}}{r^2+2}.
\end{equation}
Therefore, we arrive at
\begin{equation}
    |a(r)|\leq 4C_2\,r\,K_1(2\sqrt{2}\pi r)+2C_3\frac{e^{-\pi(r^2+2)}}{r^2+2}.
\end{equation}
It is easy to see that the left hand side of this inequality decays faster then any inverse power of $r$. Analogous estimates can be obtained for all derivatives $\frac{\dd^k}{\dd r^k}a(r)$.
\end{proof}

\begin{proposition}\label{prop:a-fourier}
\uses{lemma:Ek-Fourier, def:E2, def:a-definition, lemma:Gaussian-Fourier, prop:a-schwartz}
$a(x)$ satisfies \eqref{eqn:a-fourier}.
\end{proposition}
\begin{proof}
We recall that the Fourier transform of a Gaussian function is
\begin{equation}\label{eqn:gaussian Fourier}
    \mathcal{F}(e^{\pi i \|x\|^2 z})(y)=z^{-4}\,e^{\pi i \|y\|^2 \,(\frac{-1}{z}) }.
\end{equation}
Next, we exchange the contour integration with respect to $z$ variable and Fourier transform with respect to $x$ variable in \eqref{eqn:a-definition}.
This can be done, since the corresponding double integral converges absolutely. In this way we obtain
\begin{align}
    \widehat{a}(y)=&\int\limits_{-1}^i\phi_0\Big(\frac{-1}{z+1}\Big)\,(z+1)^2\,z^{-4}\,e^{\pi i \|y\|^2 \,(\frac{-1}{z})}\,dz
    +\int\limits_{1}^i\phi_0\Big(\frac{-1}{z-1}\Big)\,(z-1)^2\,z^{-4}\,e^{\pi i \|y\|^2 \,(\frac{-1}{z})}\,dz\notag \\
    -&2\int\limits_{0}^i\phi_0\Big(\frac{-1}{z}\Big)\,z^2\,z^{-4}\,e^{\pi i \|y\|^2 \,(\frac{-1}{z})}\,dz +2\int\limits_{i}^{i\infty}\phi_0(z)\,z^{-4}\,e^{\pi i \|y\|^2 \,(\frac{-1}{z})}\,dz.\notag
\end{align}
Now we make a change of variables $w=\frac{-1}{z}$. We obtain
\begin{align}
    \widehat{a}(y)=&\int\limits_{1}^i\phi_0\Big(1-\frac{1}{w-1}\Big)\,(\frac{-1}{w}+1)^2\,w^{2}\,e^{\pi i \|y\|^2 \,w}\,dw\notag\\
    +&\int\limits_{-1}^i\phi_0\Big(1-\frac{1}{w+1}\Big)\,(\frac{-1}{w}-1)^2\,w^2\,e^{\pi i \|y\|^2 \,w}\,dw\\
    -&2\int\limits_{i \infty}^i\phi_0(w)\,e^{\pi i \|y\|^2 \,w}\,dw +2\int\limits_{i}^{0}\phi_0\Big(\frac{-1}{w}\Big)\,w^{2}\,e^{\pi i \|y\|^2 \,w}\,dw.\notag
\end{align}
Since $\phi_0$ is $1$-periodic we have
\begin{align}
    \widehat{a}(y)\,=\,&\int\limits_{1}^i\phi_0\Big(\frac{-1}{z-1}\Big)\,(z-1)^2\,e^{\pi i \|y\|^2 \,z}\,dz
    +\int\limits_{-1}^i\phi_0\Big(\frac{-1}{z+1}\Big)\,(z+1)^2\,e^{\pi i \|y\|^2 \,z}\,dz\notag \\
    +&2\int\limits_{i}^{i\infty}\phi_0(z)\,e^{\pi i \|y\|^2 \,z}\,dz
    -2\int\limits_{0}^{i}\phi_0\Big(\frac{-1}{z}\Big)\,z^{2}\,e^{\pi i \|y\|^2 \,z}\,dz\notag \\
    \,=\,&a(y).
\end{align}
This finishes the proof of the proposition.
\end{proof}

Next, we check that $a$ has double zeroes at all $\Lambda_8$-lattice points of length greater then $\sqrt{2}$.
Using \eqref{eqn:phi0-bound}, \eqref{eqn:phi2-bound}, and \eqref{eqn:phi4-bound}, we can control the behavior of $\phi_0$ near $0$ and $i\infty$.
%Note that by definition function $a$ is radial and therefore in naturally defines a function on $\R_{\geq0}$. For abuse of notation we denote this function also by $a$.

\begin{corollary}\label{cor:phi0-near-0-infty}\uses{cor:phi0-bound, cor:phi2-bound, cor:phi4-bound, lemma:phi0-transform}
We have
\begin{align}
    \phi_0\left(\frac{i}{t}\right) &= O(e^{-2 \pi / t}) \quad \text{as } t \to 0 \label{eqn:phi0-near-0} \\
    \phi_0\left(\frac{i}{t}\right) &= O(t^{-2}e^{2 \pi t}) \quad \text{as } t \to \infty. \label{eqn:phi0-near-infty} \\
\end{align}
\end{corollary}
\begin{proof}
The first estimate follows from \eqref{eqn:phi0-bound} with $z = i/t$.
For the second estimate, by \eqref{eqn:phi0-trans-S}, \eqref{eqn:phi2-bound}, and \eqref{eqn:phi4-bound}, we have
\begin{equation}
    \left|\phi_0\left(\frac{i}{t}\right)\right| \le |\phi_0(it)| + \frac{12}{\pi t} |\phi_{-2}(it)| + \frac{36}{\pi^2 t^2} |\phi_{-4}(it)|
    \le C_0 e^{-2 \pi t} + \frac{12}{\pi t} \cdot C_{-2} + \frac{36}{\pi^2 t^2} \cdot C_{-4} e^{2 \pi t} = O(t^{-2}e^{2 \pi t}).
\end{equation}
\end{proof}

\begin{proposition}
\label{prop:a-double-zeros}
\uses{cor:phi0-near-0-infty, def:a-definition, cor:disc-nonvanishing}
For $r>\sqrt{2}$ we can express $a(r)$ in the following form
\begin{equation}\label{eqn: a double zeroes}
    a(r)=-4\sin(\pi r^2/2)^2\,\int\limits_{0}^{i\infty}\phi_0\Big(\frac{-1}{z}\Big)\,z^2\,e^{\pi i r^2 \,z}\,dz.
\end{equation}
\end{proposition}
\begin{proof}
We denote the right hand side of \eqref{eqn: a double zeroes} by $d(r)$.
Convergence of the integral for $r > \sqrt{2}$ follows from Corollary~\ref{cor:phi0-near-0-infty}.
We can write %\texttt{check signs}
\begin{align}
    d(r)=&\int\limits_{-1}^{i\infty-1}\phi_0\Big(\frac{-1}{z+1}\Big)\,(z+1)^2\,e^{\pi i r^2 \,z}\,dz-
    2\int\limits_{0}^{i\infty}\phi_0\Big(\frac{-1}{z}\Big)\,z^2\,e^{\pi i r^2 \,z}\,dz\notag\\
    +&\int\limits_{1}^{i\infty+1}\phi_0\Big(\frac{-1}{z-1}\Big)\,(z-1)^2\,e^{\pi i r^2 \,z}\,dz.\notag
\end{align}
From \eqref{eqn:phi0-trans-S} we deduce that if $r>\sqrt{2}$ then
$\phi_0\Big(\frac{-1}{z}\Big)\,z^2\,e^{\pi i r^2 \,z}\to 0$ as $\Im(z)\to\infty$. Therefore, we can deform the paths of integration
and rewrite
\begin{align}
    d(r)=&\int\limits_{-1}^{i}\phi_0\Big(\frac{-1}{z+1}\Big)\,(z+1)^2\,e^{\pi i r^2 \,z}\,dz
    +\int\limits_{i}^{i\infty}\phi_0\Big(\frac{-1}{z+1}\Big)\,(z+1)^2\,e^{\pi i r^2 \,z}\,dz\notag\\
    -2&\int\limits_{0}^{i}\phi_0\Big(\frac{-1}{z}\Big)\,z^2\,e^{\pi i r^2 \,z}\,dz
    -2\int\limits_{i}^{i\infty}\phi_0\Big(\frac{-1}{z}\Big)\,z^2\,e^{\pi i r^2 \,z}\,dz\notag\\
    +&\int\limits_{1}^{i}\phi_0\Big(\frac{-1}{z-1}\Big)\,(z-1)^2\,e^{\pi i r^2 \,z}\,dz
    +\int\limits_{i}^{i\infty}\phi_0\Big(\frac{-1}{z-1}\Big)\,(z-1)^2\,e^{\pi i r^2 \,z}\,dz.\notag
\end{align}
Now from \eqref{eqn:phi0-trans-S} we find
\begin{align}&\phi_0\Big(\frac{-1}{z+1}\Big)\,(z+1)^2-2\phi_0\Big(\frac{-1}{z}\Big)\,z^2+
\phi_0\Big(\frac{-1}{z-1}\Big)\,(z-1)^2=\notag\\
&\phi_0(z+1)\,(z+1)^2-2\phi_0(z)\,z^2+\phi_0(z-1)\,(z-1)^2\notag\\
&-\frac{12i}{\pi}\,\Big(\phi_{-2}(z+1)\,(z+1)-2\phi_{-2}(z)\,z+\phi_{-2}(z-1)\,(z-1)\Big)\notag\\
&-\frac{36}{\pi^2}\Big(\phi_{-4}(z+1)-2\phi_{-4}(z)+\phi_{-4}(z-1)\Big)=\notag\\
&2\phi_0(z).
\end{align}
Thus, we obtain
\begin{align}
    d(r)=&\int\limits_{-1}^{i}\phi_0\Big(\frac{-1}{z+1}\Big)\,(z+1)^2\,e^{\pi i r^2 \,z}\,dz
    -2\int\limits_{0}^{i}\phi_0\Big(\frac{-1}{z}\Big)\,z^2\,e^{\pi i r^2 \,z}\,dz\notag\\
    +&\int\limits_{1}^{i}\phi_0\Big(\frac{-1}{z-1}\Big)\,(z-1)^2\,e^{\pi i r^2 \,z}\,dz
    +2\int\limits_{i}^{i\infty}\phi_0(z)\,e^{\pi i r^2 \,z}\,dz=a(r).\notag
\end{align}
This finishes the proof.
\end{proof}

Finally, we find another convenient integral representation for $a$ and compute values of $a(r)$ at $r=0$ and $r=\sqrt{2}$.

\begin{proposition}\label{prop:a-another-integral}\uses{prop:a-double-zeros, def:phi4-phi2-phi0, lemma:phi0-transform, def:a-definition}
For $r\geq0$ we have
\begin{align}\label{eqn:a-another-integral}a(r)=&4i\,\sin(\pi r^2/2)^2\,\Bigg(\frac{36}{\pi^3\,(r^2-2)}-\frac{8640}{\pi^3\,r^4}+\frac{18144}{\pi^3\,r^2}\\ +&\int\limits_0^\infty\,\left(t^2\,\phi_0\Big(\frac{i}{t}\Big)-\frac{36}{\pi^2}\,e^{2\pi t}+\frac{8640}{\pi}\,t-\frac{18144}{\pi^2}\right)\,e^{-\pi r^2 t}\,dt \Bigg) .\notag\end{align}
The integral converges absolutely for all $r\in\R_{\geq 0}$.
\end{proposition}
\begin{proof}
Suppose that $r>\sqrt{2}$. Then by Proposition~\ref{prop:a-double-zeros}
$$a(r)=4i\,\sin(\pi r^2/2)^2\,\int\limits_{0}^{\infty}\phi_0(i/t)\,t^2\,e^{-\pi r^2 t}\,dt. $$
From \eqref{eqn: phi fourier0}--\eqref{eqn:phi0-trans-S} we obtain
\begin{equation}\label{eqn: phi asymptotic}
\phi_0(i/t)\,t^2=\frac{36}{\pi^2}\,e^{2 \pi t}-\frac{8640}{\pi}\,t+\frac{18144}{\pi^2}+O(t^2\,e^{-2\pi t})\quad\mbox{as}\;t\to\infty.
\end{equation}
For $r>\sqrt{2}$ we have
\begin{equation}
\int\limits_0^\infty \left(\frac{36}{\pi^2}\,e^{2 \pi t}+\frac{8640}{\pi}\,t+\frac{18144}{\pi^2}\right)\,e^{-\pi r^2 t}\,dt
=\frac{36}{\pi^3\,(r^2-2)}-\frac{8640}{\pi^3\,r^4}+\frac{18144}{\pi^3\,r^2}.\end{equation}
Therefore, the identity \eqref{eqn:a-another-integral} holds for $r>\sqrt{2}$.

On the other hand, from the definition~\eqref{eqn:a-definition} we see that $a(r)$ is analytic in some neighborhood of $[0,\infty)$. The asymptotic expansion~\eqref{eqn: phi asymptotic} implies that the right hand side of \eqref{eqn:a-another-integral} is also analytic in some neighborhood of $[0,\infty)$. Hence, the identity \eqref{eqn:a-another-integral} holds on the whole interval $[0,\infty)$. This finishes the proof of the proposition.
\end{proof}
From the identity~\eqref{eqn:a-another-integral} we see that the values $a(r)$ are in $i\R$ for all $r\in\R_{\geq0}$.
% In particular, we have
\begin{proposition}\label{prop:a0}\uses{prop:a-another-integral}
We have $a(0) = -\frac{i}{8640}$.
% \begin{equation}
% a(0)=\frac{-i\,8640}{\pi}\qquad
% a(\sqrt{2})=0\qquad
% a^\prime(\sqrt{2})=\frac{i\,72\sqrt{2}}{\pi}.
% \end{equation}
\end{proposition}
\begin{proof}
These identities follow immediately from the previous proposition.
\end{proof}

Now we construct function $b$. To this end we consider the function
\begin{definition}\label{def: h}\uses{def:H2-H3-H4}
% \begin{equation}\label{eqn: h define}
%     h(z)\,:=\,128 \frac{\theta_{00}^4(z)+\theta_{01}^4(z)}{\theta_{10}^8(z)}.
% \end{equation}
\begin{equation}\label{eqn: h define}
    h(z) := 128 \frac{H_3(z) + H_4(z)}{H_2(z)^2}.
\end{equation}
\end{definition}
It is easy to see that $h\in M^!_{-2}(\Gamma_0(2))$. Indeed, first we check that $h|_{-2}\gamma=h$
for all $\gamma\in\Gamma_0(2)$. Since the group $\Gamma_0(2)$ is generated by elements
$\left(\begin{smallmatrix}1&0\\2&1\end{smallmatrix}\right)$ and $\left(\begin{smallmatrix}1&1\\0&1\end{smallmatrix}\right)$
it suffices to check that $h$ is invariant under their action. This follows immediately
from \eqref{eqn:H2-transform-S}--\eqref{eqn:H4-transform-S} and \eqref{eqn: h define}. Next we analyze the poles of $h$.
It is known \cite[Chapter~I Lemma~4.1]{Mumford} that $\theta_{10}$ has no zeros in the upper-half plane and hence $h$ has poles only at the cusps.
At the cusp $i\infty$ this modular form has the Fourier expansion
\begin{equation}
h(z)\,=\,q^{-1} + 16 - 132 q + 640 q^2 - 2550 q^3+O(q^4).\notag
\end{equation}
Let $I=\left(\begin{smallmatrix}1&0\\0&1\end{smallmatrix}\right)$,
$T=\left(\begin{smallmatrix}1&1\\0&1\end{smallmatrix}\right)$, and
$S=\left(\begin{smallmatrix}0&-1\\1&0\end{smallmatrix}\right)$ be elements of $\Gamma_1$.
\begin{definition}\label{def:psiI-psiT-psiS}\uses{def: h}
We define the following three functions
\begin{align}
    \psi_I\,:=\,&h-h|_{-2}ST \label{eqn:psiI-define}\\
    \psi_T\,:=\,&\psi_I|_{-2}T \label{eqn:psiT-define}\\
    \psi_S\,:=\,&\psi_I|_{-2}S. \label{eqn:psiS-define}
\end{align}
\end{definition}
\begin{lemma}\label{lemma: psi I psi T psi S explicit}\uses{def:psiI-psiT-psiS}
More explicitly, we have
% \begin{align}
% \psi_I(z)\,=\,&128\,\frac{\theta_{00}^4(z)+\theta_{01}^4(z)}{\theta_{10}^8(z)}\,+\,128
%                             \frac{\theta_{01}^4(z)-\theta_{10}^4(z)}{\theta_{00}^8(z)}\label{eqn:psiI-explicit}\\
% \psi_T(z)\,=\,&128\,\frac{\theta_{00}^4(z)+\theta_{01}^4(z)}{\theta_{10}^8(z)}\,+
%                             \,128\,\frac{\theta_{00}^4(z)+\theta_{10}^4(z)}{\theta_{01}^8(z)}\label{eqn:psiT-explicit}\\
% \psi_S(z)\,=\,&-128\,\frac{\theta_{00}^4(z)+\theta_{10}^4(z)}{\theta_{01}^8(z)}-128\,
%                             \frac{\theta_{10}^4(z)-\theta_{01}^4(z)}{\theta_{00}^8(z)}.\label{eqn:psiS-explicit}
% \end{align}
\begin{align}
    \psi_I(z) &= 128 \frac{H_3(z) + H_4(z)}{H_2(z)^2} + 128 \frac{H_4(z) - H_2(z)}{H_3(z)^2} \label{eqn:psiI-explicit} \\
    \psi_T(z) &= 128 \frac{H_3(z) + H_4(z)}{H_2(z)^2} + 128 \frac{H_2(z) + H_3(z)}{H_4(z)^2} \label{eqn:psiT-explicit} \\
    \psi_S(z) &= 128 \frac{H_2(z) + H_3(z)}{H_4(z)^2} - 128 \frac{H_2(z) - H_4(z)}{H_3(z)^2} \label{eqn:psiS-explicit}
\end{align}
\end{lemma}

\begin{lemma}\label{lemma:psiI-psiT-psiS-fourier}\uses{lemma: psi I psi T psi S explicit}
The Fourier expansions of these functions are
\begin{align}
    \psi_I(z)\,=\,&q^{-1} + 144 - 5120 q^{1/2} + 70524 q - 626688 q^{3/2} + 4265600 q^2 + O(q^{5/2}) \label{eqn: psi fourier I}\\
    \psi_T(z)\,=\,&q^{-1} + 144 + 5120 q^{1/2} + 70524 q + 626688 q^{3/2} + 4265600 q^2 + O(q^{5/2}) \label{eqn: psi fourier T}\\
    \psi_S(z)\,=\,&-10240 q^{1/2} - 1253376 q^{3/2} - 48328704 q^{5/2} - 1059078144 q^{7/2}+O(q^{9/2}).\label{eqn: psi fourier S}
\end{align}
\end{lemma}
\begin{definition}\label{def:b-definition}\uses{def:psiI-psiT-psiS}
For $x\in\R^8$ define
\begin{align}\label{eqn:b-definition}
    b(x):= & \int\limits_{-1}^{i}\psi_T(z)\,e^{\pi i \|x\|^2 z}\,dz
        + \int\limits_{1}^{i}\psi_T(z)\,e^{\pi i \|x\|^2 z}\,dz \\
    -& 2\,\int\limits_{0}^{i}\psi_I(z)\,e^{\pi i \|x\|^2 z}\,dz
    - 2\,\int\limits_{i}^{i\infty}\psi_S(z)\,e^{\pi i \|x\|^2 z}\,dz \nonumber.
\end{align}
\end{definition}

Now we prove that $b$ is a Schwartz function and satisfies condition \eqref{eqn:b-fourier}.

\begin{lemma}\label{lemma:psiS-new}\uses{lemma: psi I psi T psi S explicit, lemma:lv1-lv2-identities, lemma:jacobi-identity}
$\psi_S(z)$ can be written as
\begin{equation}\label{eqn:psiS-new}
    \psi_S(z) = -\frac{H_2^3 (2 H_2^3 + 5 H_2 H_4 + 5 H_4^2)}{2 \Delta}.
\end{equation}
\end{lemma}
\begin{proof}
Using \eqref{eqn:psiS-explicit} and \eqref{eqn:jacobi-identity} gives
\begin{align}
    \psi_S &= -128 \frac{H_3 + H_2}{H_4^2} - 128 \frac{H_2 - H_4}{H_3^2} \\
    &= -128 \frac{H_3^2 (H_2 - H_4) + H_4^2 (H_2 - H_4)}{H_3^2 H_4^2} \\
    &= -128 \frac{(H_2 + H_4)^2 (2H_2 + H_4) + H_4^2(H_2 + H_4)}{H_3^2 H_4^2} \\
    &= -128 \frac{H_2 (2H_2^2 + 5 H_2 H_4 + 5 H_4^2)}{H_3^2 H_4^2} \\
    &= -128 \frac{H_2^3 (2H_2^2 + 5 H_2 H_4 + 5 H_4^2)}{H_2^2 H_3^2 H_4^2} \\
    &= - \frac{1}{2} \frac{H_2^3 (2H_2^2 + 5 H_2 H_4 + 5 H_4^2)}{\Delta}.
\end{align}
\end{proof}

\begin{lemma}\label{lemma:psiS-bound}\uses{lemma:psiS-new, lemma:mod-div-disc-bound, prop:H2-fourier, prop:H3-fourier, prop:H4-fourier}
There exists a constant $C_S > 0$ such that
\begin{equation}\label{eqn:psiS-bound}
    |\psi_S(z)| \le C_S e^{- \pi \Im z}
\end{equation}
for all $z$ with $\Im z > 1/2$.
\end{lemma}
\begin{proof}
Proof is similar to that of Lemma \ref{cor:phi0-bound}.
By Proposition \ref{prop:H2-fourier}, \ref{prop:H3-fourier} and \ref{prop:H4-fourier}, we can write Fourier expansion of the numerator of $\psi_S$ as
\begin{equation}
    H_2(z)^3 (2 H_2(z)^2 + 5 H_2(z) H_4(z) + 5 H_4(z)^2) = \sum_{n \ge 3} a_n e^{\pi i n z}
\end{equation}
with $a_3 = 16^3 \cdot 5 = 20480$ and $a_n = O(n^k)$ for some $k > 0$.
Now the result follows from Lemma \ref{lemma:mod-div-disc-bound}.
\end{proof}

\begin{proposition}\label{prop:b-schwartz}\uses{lemma:psiS-bound}
$b(x)$ is a Schwartz function.
\end{proposition}
\begin{proof}
We have
\begin{align}
    &\int\limits_{-1}^{i}\psi_T(z)\,e^{\pi i r^2 z}\,dz=\int\limits_{0}^{i+1}\psi_I(z)\,e^{\pi i r^2 (z-1)}\,dz=\notag\\
    &\int\limits_{i\infty}^{-1/(i+1)}\psi_I\Big(\frac{-1}{z}\Big)\,e^{\pi i r^2 (-1/z-1)}\,z^{-2}\,dz=\int\limits_{i\infty}^{-1/(i+1)}\psi_S(z)\,z^{-4}\,e^{\pi i r^2 (-1/z-1)}\,dz.\notag
\end{align}
Using \eqref{eqn:psiS-bound}, we can estimate the first summand in the left-hand side of~\eqref{eqn:b-definition}
\begin{equation}
    \left|\int\limits_{-1}^i \psi_T(z)\,e^{\pi i r^2 z}\,dz \right|\leq C_1\,r\,K_1(2\pi r).
\end{equation}
We combine this inequality with analogous estimates for the other three summands and obtain
\begin{equation}
    |b(r)|\leq C_2\,r\,K_1(2\pi r)+C_3\,\frac{e^{-\pi(r^2+1)}}{r^2+1}.
\end{equation}
Here $C_1$, $C_2$, and $C_3$ are some positive constants. Similar estimates hold for all derivatives $\frac{\dd^k}{\dd^k r} b(r)$.
\end{proof}

\begin{proposition}\label{prop:b-fourier}\uses{def:b-definition, lemma:Gaussian-Fourier, def:psiI-psiT-psiS, prop:b-schwartz}
$b(x)$ satisfies \eqref{eqn:b-fourier}.
\end{proposition}
\begin{proof}
Here, we repeat the arguments used in the proof of Proposition~\ref{prop:a-fourier}.
We use identity~\eqref{eqn:gaussian Fourier} and change contour integration in $z$ and Fourier transform in $x$. Thus we obtain
\begin{align}
    \mathcal{F}(b)(x)= & \int\limits_{-1}^{i}\psi_T(z)\,z^{-4}\,e^{\pi i \|x\|^2 (\frac{-1}{z})}\,dz
        + \int\limits_{1}^{i}\psi_T(z)\,z^{-4}\,e^{\pi i \|x\|^2 (\frac{-1}{z})}\,dz \notag \\
    -& 2\,\int\limits_{0}^{i}\psi_I(z)\,z^{-4}\,e^{\pi i \|x\|^2 (\frac{-1}{z})}\,dz
    - 2\,\int\limits_{i}^{i\infty}\psi_S(z)\,z^{-4}\,e^{\pi i \|x\|^2 (\frac{-1}{z})}\,dz. \notag
\end{align}
We make the change of variables $w=\frac{-1}{z}$ and arrive at
\begin{align}
    \mathcal{F}(b)(x)= & \int\limits_{1}^{i}\psi_T\Big(\frac{-1}{w}\Big)\,w^{2}\,e^{\pi i \|x\|^2 w}\,dw
        + \int\limits_{-1}^{i}\psi_T\Big(\frac{-1}{w}\Big)\,w^{2}\,e^{\pi i \|x\|^2 w}\,dw \notag \\
    -& 2\,\int\limits_{i\infty}^{i}\psi_I\Big(\frac{-1}{w}\Big)\,w^{2}\,e^{\pi i \|x\|^2 w}\,dw
    - 2\,\int\limits_{i}^{0}\psi_S\Big(\frac{-1}{w}\Big)\,w^{2}\,e^{\pi i \|x\|^2 w}\,dw.\notag
\end{align}
Now we observe that the definitions \eqref{eqn:psiI-define}--\eqref{eqn:psiS-define} imply
\begin{align}
    \psi_T|_{-2}S=&-\psi_T \notag \\
    \psi_I|_{-2}S=&\psi_S \notag \\
    \psi_S|_{-2}S=&\psi_I. \notag
\end{align}
Therefore, we arrive at
\begin{align}
    \mathcal{F}(b)(x)= & \int\limits_{1}^{i}-\psi_T(z)\,e^{\pi i \|x\|^2 z}\,dz
        + \int\limits_{-1}^{i}-\psi_T(z)\,e^{\pi i \|x\|^2 z}\,dz \notag \\
    +& 2\,\int\limits_{i}^{i\infty}\psi_S(z)\,e^{\pi i \|x\|^2 z}\,dz
    + 2\,\int\limits_{0}^{i}\psi_I(z)\,e^{\pi i \|x\|^2 w}\,dw.\notag
\end{align}
Now from~\eqref{eqn:b-definition} we see that
$$ \mathcal{F}(b)(x)=-b(x). $$
\end{proof}
Now we regard the radial function $b$ as a function on $\R_{\geq0}$. We check that $b$ has double roots at $\Lambda_8$-points.

\begin{lemma}\label{lemma:psiI-bound}\uses{lemma:mod-div-disc-bound, prop:H2-fourier, prop:H3-fourier, prop:H4-fourier, def:psiI-psiT-psiS, lemma:theta-transform-S-T, lemma:disc-cuspform}
There exists a constant $C_I > 0$ such that
\begin{equation}
    |\psi_I(z)| \le C_I e^{2 \pi \Im z}
\end{equation}
for all $z$ with $\Im z > 1/2$.
\end{lemma}
\begin{proof}
By \eqref{eqn:psiS-new}, \eqref{eqn:psiS-define}, \eqref{eqn:H2-transform-S}, and \eqref{eqn:H4-transform-S},
\begin{equation}
    \psi_I(z) = \frac{H_4^3(2 H_4^2 + 5 H_4 H_2 + 5 H_2^2)}{2 \Delta}.
\end{equation}
The denominator is not a cusp form (i.e. has a nonzero constant term), hence Lemma \ref{lemma:mod-div-disc-bound} concludes the proof with $n_0 = 0$.
\end{proof}

\begin{corollary}\label{cor:psiI-near-0-infty}\uses{lemma:psiI-bound, lemma:psiS-bound, def:psiI-psiT-psiS}
We have
\begin{align}
    \psi_I(it) &= O(t^2 e^{\pi/t}) \quad \text{as } t \to 0 \label{eqn:psiI-near-0} \\
    \psi_I(it) &= O(e^{2 \pi t}) \quad \text{as } t \to \infty. \label{eqn:psiI-near-infty}
\end{align}
\end{corollary}
\begin{proof}
By \eqref{eqn:psiS-define}, we have
\begin{equation}
    \psi_I(it) = (it)^{-2} \psi_S\left(\frac{-1}{it}\right) = -t^{-2} \psi_S\left(\frac{i}{t}\right).
\end{equation}
and combined with \eqref{eqn:psiS-bound} we get \eqref{eqn:psiI-near-0}.
\eqref{eqn:psiI-near-infty} follows from Lemma \ref{lemma:psiI-bound}.
\end{proof}

\begin{proposition}\label{prop:b-double-zeros}\uses{lemma:psiI-psiT-psiS-fourier, def:psiI-psiT-psiS, cor:psiI-near-0-infty, cor:disc-nonvanishing}
For $r>\sqrt{2}$ function $b(r)$ can be expressed as
\begin{equation}\label{eqn: b double zeroes}
    b(r)=-4\sin(\pi r^2/2)^2\,\int\limits_{0}^{i\infty}\psi_I(z)\,e^{\pi i r^2 \,z}\,dz.
\end{equation}
\end{proposition}
\begin{proof}
We denote the right hand side of~\eqref{eqn: b double zeroes} by $c(r)$.
By Corollary \ref{cor:psiI-near-0-infty}, the integral in~\eqref{eqn: b double zeroes} converges for $r>\sqrt{2}$.
Then we rewrite it in the following way:
$$c(r)=\int\limits_{-1}^{i\infty-1}\psi_I(z+1)\,e^{\pi i r^2 \,z}\,dz-2\int\limits_{0}^{i\infty}\psi_I(z)\,e^{\pi i r^2 \,z}\,dz+
\int\limits_{1}^{i\infty+1}\psi_I(z-1)\,e^{\pi i r^2 \,z}\,dz.$$
From the Fourier expansion~\eqref{eqn: psi fourier I} we know that $\psi_I(z)=e^{-2\pi i z}+O(1)$ as $\Im(z)\to\infty$.
By assumption $r^2>2$, hence we can deform the path of integration and write
\begin{align}\label{eqn: inside proof 1}
\int\limits_{-1}^{i\infty-1}\psi_I(z+1)\,e^{\pi i r^2 \,z}\,dz=&
\int\limits_{-1}^{i}\psi_T(z)\,e^{\pi i r^2 \,z}\,dz+\int\limits_{i}^{i\infty}\psi_T(z)\,e^{\pi i r^2 \,z}\,dz\\
\int\limits_{1}^{i\infty+1}\psi_I(z-1)\,e^{\pi i r^2 \,z}\,dz=&
\int\limits_{-1}^{i}\psi_T(z)\,e^{\pi i r^2 \,z}\,dz+\int\limits_{i}^{i\infty}\psi_T(z)\,e^{\pi i r^2 \,z}\,dz.
\end{align}
We have
\begin{align}\label{eqn: c1}c(r)=&\int\limits_{-1}^{i}\psi_T(z)\,e^{\pi i r^2 \,z}\,dz+\int\limits_{1}^{i}\psi_T(z)\,e^{\pi i r^2 \,z}\,dz
-2\int\limits_{0}^{i}\psi_I(z)\,e^{\pi i r^2 \,z}\,dz\\
&+2\int\limits_{i}^{i\infty}(\psi_T(z)-\psi_I(z))\,e^{\pi i r^2 \,z}\,dz.\nonumber
    \end{align}
Next, we check that the functions $\psi_I,\psi_T$, and $\psi_S$ satisfy the following identity:
\begin{equation}\label{eqn: c2}\psi_T+\psi_S=\psi_I.\end{equation}
Indeed, from definitions \eqref{eqn:psiI-define}-\eqref{eqn:psiS-define} we get
\begin{align}\psi_T+\psi_S=&(h-h|_{-2}ST)|_{-2}T+(h-h|_{-2}ST)|_{-2}S\notag\\
=&h|_{-2}T-h|_{-2}ST^2+h|_{-2}S-h|_{-2}STS.\notag\end{align}
Note that $ST^2S$ belongs to $\Gamma_0(2)$. Thus, since $h\in M^!_{-2}\Gamma_0(2)$ we get
$$\psi_T+\psi_S=h|_{-2}T-h|_{-2}STS. $$
Now we observe that $T$ and $STS(ST)^{-1}$ are also in $\Gamma_0(2)$. Therefore,
$$\psi_T+\psi_S=h|_{-2}T-h|_{-2}STS=h|_{-2}-h|ST=\psi_I.$$

Combining \eqref{eqn: c1} and \eqref{eqn: c2} we find
\begin{align}c(r)=&\int\limits_{-1}^{i}\psi_T(z)\,e^{\pi i r^2 \,z}\,dz+\int\limits_{1}^{i}\psi_T(z)\,e^{\pi i r^2 \,z}\,dz
-2\int\limits_{0}^{i}\psi_I(z)\,e^{\pi i r^2 \,z}\,dz\notag\\
&-2\int\limits_{i}^{i\infty}\psi_S(z)\,e^{\pi i r^2 \,z}\,dz\notag\\
=&b(r).\notag
    \end{align}
\end{proof}
At the end of this section we find another integral representation of $b(r)$ for $r\in\R_{\geq0}$ and compute special values of $b$.

\begin{proposition}\label{prop:b-another-integral}\uses{prop:b-double-zeros, lemma:psiI-psiT-psiS-fourier, def:b-definition, eqn: psi asymptotic}
For $r\geq0$ we have
\begin{equation}\label{eqn:b-another-integral}b(r)=4i\,\sin(\pi r^2/2)^2\,\left(\frac{144}{\pi\,r^2}+\frac{1}{\pi\,(r^2-2)}+\int\limits_0^\infty\,\left(\psi_I(it)-144-e^{2\pi t}\right)\,e^{-\pi r^2 t}\,dt\right).\end{equation}
The integral converges absolutely for all $r\in\R_{\geq 0}$.
\end{proposition}
\begin{proof}
The proof is analogous to the proof of Proposition~\ref{prop:a-another-integral}.
First, suppose that $r>\sqrt{2}$. Then by Proposition~\ref{prop:b-double-zeros}
$$b(r)=4i\,\sin(\pi r^2/2)^2\,\int\limits_{0}^{\infty}\psi_I(it)\,e^{-\pi r^2 t}\,dt. $$
From \eqref{eqn: psi fourier I} we obtain
\begin{equation}\label{eqn: psi asymptotic}
\psi_I(it)=e^{2\pi t}+144+O(e^{-\pi t})\quad\mbox{as}\;t\to\infty.
\end{equation}
For $r>\sqrt{2}$ we have
\begin{equation}
\int\limits_0^\infty \left(e^{2\pi t}+144\right)\,e^{-\pi r^2 t}\,dt
=\frac{1}{\pi\,(r^2-2)}+\frac{144}{\pi\,r^2}.\end{equation}
Therefore, the identity \eqref{eqn:b-another-integral} holds for $r>\sqrt{2}$.

On the other hand, from the definition \eqref{eqn:b-definition} we see that $b(r)$ is analytic in some neighborhood of $[0,\infty)$. The asymptotic expansion \eqref{eqn: psi asymptotic} implies that the right hand side of \eqref{eqn:b-another-integral} is also analytic in some neighborhood of $[0,\infty)$. Hence, the identity \eqref{eqn:b-another-integral} holds on the whole interval $[0,\infty)$. This finishes the proof of the proposition.
\end{proof}
We see from \eqref{eqn:b-another-integral} that $b(r)\in i\R$ far all $r\in\R_\geq{0}$.
Another immediate corollary of this proposition is

\begin{proposition}\label{prop:b0}\uses{prop:b-another-integral}
We have $b(0) = 0$.
% \begin{equation}\label{eqn: b values}
% b(0)=0\qquad
% % b(\sqrt{2})=0\qquad
% % b^\prime(\sqrt{2})=\frac{i}{2\sqrt{2}\,\pi}.
% \end{equation}
\end{proposition}
