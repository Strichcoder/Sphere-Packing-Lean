% Sidharth: A lot of this might exist already. We can clear things up quite easily.
% Gareth: lol, not quite

In this section, we recall and develop some theory of (quasi)modular forms.

\subsection{Modular forms and examples}

Let $\h$ be the upper half-plane $\{z\in\C\mid\Im(z)>0\}$.
\begin{lemma}\label{def:Gamma-1-Action}
    The modular group $\Gamma_1:=\mathrm{SL}_2(\Z)$ acts on $\h$ by linear fractional transformations
$$\left(\begin{smallmatrix}a&b\\c&d\end{smallmatrix}\right)z:=\frac{az+b}{cz+d}.$$
\end{lemma}

Let $N$ be a positive integer.
\begin{definition}\label{def:level-N-princ-cong-subgp}
    The \emph{level $N$ principal congruence subgroup} of $\Gamma_1$ is
    $$\Gamma(N):=\left\{\left.\left(\begin{smallmatrix}a&b\\c&d\end{smallmatrix}\right)\in\Gamma_1\right|\left(\begin{smallmatrix}a&b\\c&d\end{smallmatrix}\right)\equiv\left(\begin{smallmatrix}1&0\\0&1\end{smallmatrix}\right)\;\mathrm{mod}\;N\right\}.$$
\end{definition}

\begin{definition}\label{def:congruence-subgroup}\uses{def:level-N-princ-cong-subgp}
    A subgroup $\Gamma\subset\Gamma_1$ is called a \emph{congruence subgroup} if $\Gamma(N)\subset\Gamma$ for some $N\in\N$.
\end{definition}

\begin{definition}\label{def:Gamma-generators}\uses{def:level-N-princ-cong-subgp}\lean{ModularGroup.S,ModularGroup.T,α,β}\leanok
  Define the matrices

  \[
    S = \begin{pmatrix} 0 & -1 \\ 1 & 0 \end{pmatrix} \in \Gamma_1,
    T = \begin{pmatrix} 1 & 1 \\ 0 & 1 \end{pmatrix} \in \Gamma_1,
    \alpha = \begin{pmatrix} 1 & 2 \\ 0 & 1 \end{pmatrix} \in \Gamma_2 \subset \Gamma_1,
    \beta = \begin{pmatrix} 1 & 0 \\ 2 & 1 \end{pmatrix} \in \Gamma_2 \subset \Gamma_1.
  \]

  It is easily verifiable that $\alpha = T^2$ and $\beta = -S\alpha^{-1}S = -ST^{-2}S$.
\end{definition}

The following two lemmas tell us the group structure of $\Gamma(1) = \Gamma_1$ and $\Gamma(2)$, which we will use later on to define the theta forms.

\begin{lemma}\label{lemma:Gamma-1-generators}\uses{def:Gamma-generators}\lean{SL2Z_generate}
  We have $\Gamma(1) = \langle S, T, -I \rangle$.
\end{lemma}
\begin{proof}
  See~\cite[Exercise 1.1.1]{first course}.
\end{proof}

\begin{lemma}\label{lemma:Gamma-2-generators}\uses{def:Gamma-generators}\lean{Γ2_generate}
  We have $\Gamma(2) = \langle \alpha, \gamma, -I \rangle$.
\end{lemma}
\begin{proof}
  See~\cite[Exercise 1.2.4]{first course}.
\end{proof}

Let $z\in\h$, $k\in\Z$, and $\left(\begin{smallmatrix}a&b\\c&d\end{smallmatrix}\right)\in\mathrm{SL}_2(\Z)$. We omit many of the proofs below when they exist in Mathlib already.
\begin{definition}\label{def:automorphy-factor}
    The \emph{automorphy factor} of weight $k$ is defined as
$$j_k(z,\left(\begin{smallmatrix}a&b\\c&d\end{smallmatrix}\right)):=(cz+d)^{-k}.$$
\end{definition}

\begin{lemma}\label{lemma:automorphy-factor-chain-rule}\uses{def:automorphy-factor}\leanok
    The automorphy factor satisfies the \emph{chain rule}
$$j_k(z,\gamma_1\gamma_2)=j_k(z,\gamma_1)\,j_k(\gamma_2z,\gamma_1). $$
\end{lemma}

\begin{definition}\label{def:slash-operator}\uses{def:automorphy-factor}\leanok
    Let $F$ be a function on $\h$ and $\gamma\in\mathrm{SL}_2(\Z)$. Then the \emph{slash operator} acts on $F$ by
$$(F|_k\gamma)(z):=j_k(z,\gamma)\,F(\gamma z). $$
\end{definition}

\begin{lemma}\label{lemma:slash-operator-chain-rule}\uses{lemma:automorphy-factor-chain-rule}\lean{SlashAction.slash_mul}\leanok
  The chain rule implies $$F|_k\gamma_1\gamma_2=(F|_k\gamma_1)|_k\gamma_2.$$
\end{lemma}

In particular, this lemma implies that if $\Gamma = \langle M_i \rangle_{i \in \mathcal{I}}$, then the slash action $F|\gamma$ is uniquely determined by the action of generators, i.e. $F|M_i$ and $F|M_i^{-1}$.

\begin{lemma}\label{lemma:slash-negI-even-weight}\uses{def:slash-operator}\lean{modular_slash_negI_of_even}\leanok
   For even $k$, $F|_{k}(-I) = F$.
\end{lemma}
\begin{proof}
Follows from the definition of the slash operator:
$(F|_{k}(-I))(z) = (-1)^{-k}F((-I)z) = F(z)$.
\end{proof}

\begin{definition}\label{def:holomorphic-modular-form}\uses{def:congruence-subgroup}\lean{ModularForm}\leanok
A \emph{(holomorphic) modular form} of integer weight $k$ and congruence subgroup $\Gamma$ is a holomorphic function $f:\h\to\C$ such that:
\begin{enumerate}
  \item{(Slash invariant)} $f|_k\gamma=f$ for all $\gamma\in\Gamma$
  \item{(Holomorphic at $i\infty$)} for each $\alpha\in\Gamma_1\;f|_k\alpha$ has the Fourier expansion $f|_k\alpha (z)=\sum_{n=0}^\infty c_f(\alpha,\frac{n}{n_\alpha})\,e^{2\pi i \frac{n}{n_\alpha}z}$ for some $n_\alpha\in\N$ and Fourier coefficients $c_f(\alpha,m)\in\C$.
\end{enumerate}
\end{definition}

\begin{definition}\label{def:Mk}\uses{def:holomorphic-modular-form}\lean{ModularForm}\leanok
    Let $M_k(\Gamma)$ be the space of modular forms of weight $k$ and congruence subgroup $\Gamma$.
\end{definition}

A key fact in the theory of modular forms is the following theorem:
\begin{theorem}\label{theorem-Mk-finite-dimensional}\uses{def:Mk}
    The spaces $M_k(\Gamma)$ are finite dimensional.
\end{theorem}
\begin{proof}
  In this project, we only require the theorem for $\Gamma = \Gamma_1$ and $\Gamma = \Gamma_2$. For the proof, see~\cite{Serre73}.
\end{proof}

Let us consider several examples of modular forms.
\begin{definition}\label{def:Ek-definition}% \lean{def:Ek-definition}
For an even integer $k\geq 4$ we define the \emph{weight $k$ Eisenstein series} as
\begin{equation}\label{eqn:Ek-definition}
E_k(z):=\frac{1}{2\zeta(k)}\sum_{(c,d)\in\Z^2\backslash(0,0)}(cz+d)^{-k}.\end{equation}
\end{definition}
\begin{lemma}\label{lemma:Ek-is-modular-form}\uses{def:Mk}
For all $k$, $E_k\in M_k(\Gamma_1)$.
Especially, we have
\begin{equation}\label{eqn:Ek-trans-S}
    E_k \left(-\frac{1}{z}\right) = z^k E_k(z).
\end{equation}
\end{lemma}
\begin{proof}
This follows from the fact that the sum converges absolutely.
Now apply slash operator with $\gamma = \left(\begin{smallmatrix} 0 & -1 \\ 1 & 0 \end{smallmatrix}\right)$ gives \eqref{eqn:Ek-trans-S}.
\end{proof}

\begin{lemma}\label{lemma:Ek-Fourier}\uses{def:Ek-definition}
% \lean{lemma:Ek-Fourier}\uses{def:Ek-definition}
The Eisenstein series possesses the Fourier expansion
\begin{equation}\label{eqn:Ek-Fourier}E_k(z)=1+\frac{2}{\zeta(1-k)}\sum_{n=1}^\infty \sigma_{k-1}(n)\,e^{2\pi i z}, \end{equation}
where $\sigma_{k-1}(n)\,=\,\sum_{d|n} d^{k-1}$. In particular, we have
\begin{align}
  E_4(z)\,=\,& 1+240\sum_{n=1}^\infty \sigma_3(n)\,e^{2\pi i n z} \notag \\
  E_6(z)\,=\,& 1-504\sum_{n=1}^\infty \sigma_5(n)\,e^{2\pi i n z}. \notag
\end{align}
\end{lemma}
The infinite sum \eqref{eqn:Ek-definition} does not converge absolutely for $k=2$.
On the other hand, the expression \eqref{eqn:Ek-Fourier} converges to a holomorphic function on the upper half-plane and we will take it as a definition of $E_2$ (See Definition \ref{def:E2} below).


The discriminant form is a unique normalized cusp form of weight 12, which can be defined using $E_4$ and $E_6$.
\begin{definition}\label{def:disc-definition}% \lean{def:disc-definition}
The \emph{discriminant form} $\Delta(z)$ is given by
\begin{equation}\label{eqn:disc-definition}
\Delta(z) = \frac{E_4(z)^3 - E_6(z)^2}{1728}
\end{equation}
\end{definition}

\begin{lemma}\label{lemma:disc-cuspform}\uses{def:disc-definition}
$\Delta(z) \in M_{12}(\Gamma_1)$ and it vanishes at the unique cusp, i.e. it is a cusp form of level $\Gamma_1$ and weight $12$.
\end{lemma}
\begin{proof}
Being a modular form of desired weight and level directly follows from those of $E_4$ and $E_6$.
It is a cusp form since the constant terms of Fourier expansions of $E_4$ and $E_6$ are both $1$.
\end{proof}

It also admits a product formula, which allow us to prove positivity of $\Delta(it)$ for $t > 0$ later.
\begin{lemma}\label{lemma:disc-prodformula}\uses{def:disc-definition}
We have
\begin{equation}\label{eqn:disc-prodformula}
\Delta(z) = e^{2 \pi i z} \prod_{n \ge 1} (1 - e^{2 \pi i n z})^{24}.
\end{equation}
\end{lemma}
\begin{proof}
There are several known proofs of this.
One possible proof that we can formalize is from Kohnen \cite{Kohnen}, which prove
\begin{equation}\label{eqn:disc-logder}
    \frac{1}{2\pi i z} \frac{d}{dz}(\log \Delta) = 1 - 24 \sum_{n \ge 1} \frac{ne^{2 \pi i n z}}{1 - e^{2 \pi i n z}}.
\end{equation}
by using a multiplicative analogue of the Hecke operator and the valence formula.
% Here we briefly summarize the proof.
% First of all, it is enough to prove that the logarithmic derivative of $\Delta$ is given by
% \begin{equation}\label{eqn:disc-logder}
%     \frac{1}{2\pi i z} \frac{d}{dz}(\log \Delta) = 1 - 24 \sum_{n \ge 1} \frac{ne^{2 \pi i n z}}{1 - e^{2 \pi i n z}}.
% \end{equation}
% Define the slash action of $\gamma = \left(\begin{smallmatrix}a & b \\ c & d\end{smallmatrix}\right) \in \mathrm{GL}_{2}^{+}(\R)$ as
% \begin{equation}\label{eqn:slash-operator-gl2p}
% (f|_{k} \gamma)(z) := (\det \gamma)^{k/2} j_k(z, \gamma) f(\gamma z).
% \end{equation}
% The ``multiplicative'' Hecke operator $T_{m}^{M}$ of weight $m \ge 1$ is given by
% \begin{equation}\label{eqn:mult-hecke}
% T_{m}^{M}(f) := \prod_{\gamma \in \Gamma_1 \backslash \mathcal{M}(m)} (f|_{m}\gamma)(z)
% \end{equation}
% where
% \begin{equation}\label{eqn:matm}
% \mathcal{M}(m) := \{\gamma in M_{2}(\Z): \det(\gamma) = m\} = \coprod_{\substack{ad = m, d > 0 \\ b\,(\mathrm{mod}\, d)}} \Gamma_1 \begin{pmatrix}a & b \\ 0 & d\end{pmatrix}.
% \end{equation}
% Then $\# \Gamma_1 \backslash \mathcal{M}(m) = \sigma_1(m)$, and for $f \in M_{k}(\Gamma_1)$, we have $T_{m}^{M}(f) \in M_{k \sigma_1(m)}(\Gamma_1)$.
\end{proof}

Note that the RHS of \eqref{eqn:disc-logder} is equal to the $E_2(z)$.
As a side note, we can also consider defining $\Delta$ as \eqref{eqn:disc-prodformula}, and prove that it coincides with \eqref{eqn:disc-definition}.
Such an argument can be found in \cite[Section 2.4]{Bruinier}.

\begin{corollary}\label{cor:disc-pos}\uses{lemma:disc-prodformula}
$\Delta(it) > 0$ for all $t > 0$.
\end{corollary}
\begin{proof}
By Lemma \ref{lemma:disc-prodformula}, we have
$$
\Delta(it) = e^{-2 \pi t} \prod_{n \ge 1} (1 - e^{-2 \pi n t})^{24} > 0.
$$
\end{proof}

Another examples of modular forms we would like to consider are \emph{theta functions} \cite[Section~3.1]{1-2-3}.
\begin{definition}\label{def:th00-th01-th10}
We define three different theta functions (so called ``Thetanullwerte'') as
\begin{align}
  \Theta_{2}(z) = \theta_{10}(z)\,=\, & \sum_{n\in\Z}e^{\pi i (n+\frac12)^2 z}. \notag \\
  \Theta_{3}(z) = \theta_{00}(z)\,=\, & \sum_{n\in\Z}e^{\pi i n^2 z} \notag \\
  \Theta_{4}(z) = \theta_{01}(z)\,=\, & \sum_{n\in\Z}(-1)^n\,e^{\pi i n^2 z} \notag \\
\end{align}
\end{definition}

For convenience, we use the following notations for the fourth powers of the theta functions.
\begin{definition}\label{def:H2-H3-H4}\uses{def:th00-th01-th10}
Define
\begin{equation}
    H_2 = \Theta_2^4, \quad H_3 = \Theta_3^4, \quad H_4 = \Theta_4^4. \label{eqn:H2-H3-H4}
\end{equation}
\end{definition}
Note that we only need these fourth powers to define \eqref{def: b(r) definition}.

The group $\Gamma_1$ is generated by the elements $T=\left(\begin{smallmatrix}1&1\\0&1\end{smallmatrix}\right)$, $S=\left(\begin{smallmatrix}0&1\\-1&0\end{smallmatrix}\right)$, and $-I = \left(\begin{smallmatrix}-1&0\\0&-1\end{smallmatrix}\right)$ (\cref{lemma:Gamma-1-generators}), and the transformation of functions under $\Gamma(2)$ is determined by that under $\Gamma_1$ (by \cref{lemma:slash-operator-chain-rule}). The following lemma shows how the theta functions (and their powers) transform under the slash action of these matrices.

\begin{lemma}\label{lemma:theta-transform-S-T}\uses{def:th00-th01-th10, def:H2-H3-H4}\lean{H₂_T_action,H₃_T_action,H₄_T_action,H₂_S_action,H₃_S_action,H₄_S_action}\leanok
These elements act on the theta functions in the following way
% \begin{align}
% z^{-2}\,\theta^4_{00}\Big(\frac{-1}{z}\Big)\,=\,&-\theta_{00}^4(z) \label{eqn: theta transform S}\\
% z^{-2}\,\theta^4_{01}\Big(\frac{-1}{z}\Big)\,=\,&-\theta_{10}^4(z) \label{eqn: theta01 transform S}\\
% z^{-2}\,\theta^4_{10}\Big(\frac{-1}{z}\Big)\,=\,&-\theta_{01}^4(z) \label{eqn: theta10 transform S}
% \end{align}
\begin{align}
    H_2 | S &= -H_4 \label{eqn:H2-transform-S} \\
    H_3 | S &= -H_3 \label{eqn:H3-transform-S} \\
    H_4 | S &= -H_2 \label{eqn:H4-transform-S}
\end{align}
and
% \begin{align}
% \theta^4_{00}(z+1)\,=\,&\theta_{01}^4(z) \label{eqn: theta10 transform T}\\
% \theta^4_{01}(z+1)\,=\,&\theta_{00}^4(z) \label{eqn: theta01 transform T} \\
% \theta^4_{10}(z+1)\,=\,&-\theta_{10}^4(z). \label{eqn: theta transform T}
% \end{align}
\begin{align}
    H_2 | T &= -H_2 \label{eqn:H2-transform-T} \\
    H_3 | T &= H_4 \label{eqn:H3-transform-T} \\
    H_4 | T &= H_3 \label{eqn:H4-transform-T}
\end{align}
\end{lemma}
\begin{proof}\leanok
The last three identities easily follow from the definition.
For example, \eqref{eqn:H2-transform-T} follows from
\begin{align}
    \Theta_{2}(z + 1) &= \sum_{n\in\Z}e^{\pi i (n+\frac12)^2 (z + 1)}
    = \sum_{n \in \Z} e^{\pi i (n + \frac{1}{2})^{2}} e^{\pi i (n + \frac{1}{2})^{2} z} \\
    &= \sum_{n \in \Z} e^{\pi i (n^2 + n + \frac{1}{4})} e^{\pi i (n + \frac{1}{2})^{2} z} = \sum_{n \in \Z} (-1)^{n^2 + n}e^{\pi i / 4} e^{\pi i (n + \frac{1}{2})^{2} z} \\
    &= e^{\pi i / 4} \Theta_{2}(z)
\end{align}
and taking 4th power.
\eqref{eqn:H2-transform-S} and \eqref{eqn:H4-transform-S} are equivalent under $z \leftrightarrow -1/z$, so it is enough to show \eqref{eqn:H2-transform-S} and \eqref{eqn:H3-transform-S}.
These identities follow from the identities of the \emph{two-variable} Jacobi theta function, which is defined as (be careful for the variables, where we use $\tau$ instead of $z$)
\begin{equation}
    \theta(z, \tau) = \sum_{n \in \mathbb{Z}} e^{2 \pi i n z + \pi i n^2 \tau} \label{eqn:jacobi2}
\end{equation}
and already formalized by David Loeffler.
This function specialize to the theta functions as
\begin{align}
    \Theta_{2}(\tau) &= e^{\pi i \tau / 4} \theta(-\tau / 2, \tau) \label{eqn:Th2-as-jacobi2} \\
    \Theta_{3}(\tau) &= \theta(0, \tau) \label{eqn:Th3-as-jacobi2} \\
    \Theta_{4}(\tau) &= \theta(1/2, \tau) \label{eqn:Th4-as-jacobi2} \\
\end{align}
Possion summation formula gives
\begin{equation}
    \theta(z, \tau) = \frac{1}{\sqrt{-i \tau}} e^{-\frac{\pi i z^2}{\tau}} \theta\left(\frac{z}{\tau}, -\frac{1}{\tau}\right) \label{eqn:jacobi2transform}
\end{equation}
and applying the specializations above yield the identities.
For example, \eqref{eqn:H4-transform-S} follows from
\begin{equation}
    \Theta_{4}(\tau) = \theta\left(\frac{1}{2}, \tau\right) = \frac{1}{\sqrt{-i\tau}} e^{- \frac{\pi i }{4 \tau}} \theta\left(\frac{1}{2 \tau}, -\frac{1}{\tau}\right) = \frac{1}{\sqrt{-i\tau}} \Theta_{2}\left(-\frac{1}{\tau}\right)
\end{equation}
and taking 4th power.
\end{proof}

Using the above identities, we can prove that these are modular forms.
\begin{lemma}\label{lemma:theta-slash-invariant}\uses{lemma:slash-operator-chain-rule,lemma:slash-negI-even-weight,lemma:theta-transform-S-T}\lean{H₂_SIF,H₃_SIF,H₄_SIF}\leanok
  $H_{2}$, $H_{3}$, and $H_{4}$ are slash invariant under $\Gamma(2)$, i.e. for all $\gamma \in \Gamma(2)$ and $i \in \{2, 3, 4\}$, we have $H_i|\gamma = H_i|\gamma^{-1} = H_i$.
\end{lemma}
\begin{proof}\leanok
  By \cref{lemma:Gamma-2-generators} and \cref{lemma:slash-operator-chain-rule}, it suffices to show that the $H_i$ are invariant under slash actions with respect to $\alpha$, $\beta$, and $-I$.
Invariance under $-I$ follows from Lemma \ref{lemma:slash-negI-even-weight}.
The rest follows from Lemma \ref{lemma:slash-operator-chain-rule}, \ref{lemma:theta-transform-S-T}, and the matrix identities
\begin{equation}
    \alpha = T^2, \quad \beta = -S\alpha^{-1}S = -ST^{-2}S. \label{eqn:matrix}
\end{equation}
For example, invariance for $H_2$ can be proved by
\begin{align}
    H_2|\alpha &= H_2 |T^{2} = -H_2 |T = H_2 \\
    H_2|\beta &= H_2 |(-S\alpha^{-1}S) = H_2 | (S\alpha^{-1}S) =-H_4 |(\alpha^{-1}S) = -H_4 |S  = H_2.
\end{align}
\end{proof}

\begin{lemma}\label{lemma:theta-modular}\uses{lemma:theta-slash-invariant}\lean{H₂_MF,H₃_MF,H₄_MF}
$H_{2}$, $H_{3}$, and $H_{4}$ belong to $M_2(\Gamma(2))$.
\end{lemma}
\begin{proof}
  We have to show that each $H_i$ satisfies property (2) of \cref{def:holomorphic-modular-form}. It suffices to show that for any $\gamma \in \Gamma_1$, $\|H_2|_2\gamma(z)\|$ is bounded as $z \in \mathbb{H} \to i\infty$. Write $\gamma = \begin{pmatrix} a & b \\ c & d \end{pmatrix}$ with $a, b, c, d \in \Z$, and suppose that $z \in \mathbb{H}$ with $\Im(z) \geq A \in \R_{> 0}$. Then,

\begin{align}
  \|H_2|_2\gamma(z)\|
  &= \|(cz + d)^{-2} H_2(\gamma(z))\| \\
  &\leq |(cz + d)^{-2}| \|H_2(\gamma(z))\| \\
  &= \|cz + d\|^{-2} \left\|\sum_{n \in \Z} \exp\left(\pi i \left(n + \frac{1}{2}\right)^2 \gamma(z)\right)\right\|^4 \\
  &\leq \|cz + d\|^{-2} \left(\sum_{n \in \Z} \left\|\exp\left(\pi i \left(n + \frac{1}{2}\right)^2 \gamma(z)\right)\right\|\right)^4 \\
  &= \|cz + d\|^{-2} \left(\sum_{n \in \Z} \exp\left(-\pi \left(n + \frac{1}{2}\right)^2 \Im(\gamma(z))\right)\right)^4 \\
  &= \|cz + d\|^{-2} \left(\sum_{n \in \Z} \exp\left(-\pi \left(n + \frac{1}{2}\right)^2 \cdot \frac{\Im(z)}{c^2\|z\|^2 + d^2}\right)\right)^4 \\
  &\leq \|cz + d\|^{-2} \left(\sum_{n \in \Z} \exp\left(-\pi \left(n + \frac{1}{2}\right)^2 \cdot \frac{A}{c^2\|z\|^2 + d^2}\right)\right)^4 \\
\end{align}

Formalised:

\begin{align}
  a \neq 0 \implies \exists p \in \R_{> 0}, \sum_{n \in \Z} e^{-\pi (n + a)^2 t} = O_t(e^{-pt})
\end{align}

  \todo{Differentiability proof.}
\end{proof}

We also have a nontrivial relation between these theta functions.
\begin{lemma}\label{lemma:jacobi-identity}\uses{lemma:theta-modular}
These three theta functions satisfy the \emph{Jacobi identity}
\begin{equation}\label{eqn:jacobi-identity}
H_{2} + H_{4} = H_{3} \Leftrightarrow \Theta_{2}^4+ \Theta_{4}^4= \Theta_{3}^4.
\end{equation}
\end{lemma}
\begin{proof}
Use the dimesion formula of the space of modular forms of weight 2 and level $\Gamma(2)$.
Especially, we have $\dim M_2(\Gamma(2)) = 2$ with basis $H_{2}$ and $H_{4}$.
\end{proof}

These are also related to $E_4$, $E_6$, and $\Delta$ as follows.
\begin{lemma}\label{lemma:lv1-lv2-identities}\uses{lemma:theta-transform-S-T, lemma:theta-modular, lemma:disc-cuspform}
We have
\begin{align}
    E_4 &= \frac{1}{2}(H_{2}^{2} + H_{3}^{2} + H_{4}^{2}) = H_{2}^{2} + H_{2}H_{4} + H_{4}^{2} \label{eqn:e4theta} \\
    E_6 &= \frac{1}{2} (H_{2} + H_{3})(H_{3} + H_{4}) (H_{4} - H_{2}) = \frac{1}{2}(H_2 + 2H_4)(2H_2 + H_4)(H_4 - H_2) \label{eqn:e6theta} \\
    \Delta &= \frac{1}{256} (H_{2}H_{3}H_{4})^2. \label{eqn:disctheta}
\end{align}
\end{lemma}
\begin{proof}
We can prove these similarly as Lemma \ref{lemma:jacobi-identity}.
Right hand sides of \eqref{eqn:e4theta}, \eqref{eqn:e6theta}, and \eqref{eqn:disctheta} are all modular forms of level $\Gamma_1$ and desired weights, where \eqref{eqn:disctheta} is a cusp form since $H_2$ is.
Now the identities follow from the dimension calculations $\dim M_4(\Gamma_1) = \dim M_6(\Gamma_1) = \dim S_{12}(\Gamma_1) = 1$ and comparing the first nonzero $q$-coefficients.
\end{proof}

The \emph{strict} positivity of Jacobi theta functions might needed later.
\begin{lemma}\label{lemma:theta-pos}\uses{lemma:jacobi-identity, lemma:theta-transform-S-T}
All three functions $t \mapsto H_2(it), H_3(it), H_4(it)$ are positive for $t > 0$.
\end{lemma}
\begin{proof}
By Lemma \ref{lemma:jacobi-identity} and the transformation law \eqref{eqn:H2-transform-S}, it is enough to prove the positivity for $\Theta_2(it)$, which is clear from its definition:
\begin{equation}
    \Theta_{2}(it) = \sum_{n \in \Z} e^{- \pi (n + \frac{1}{2})^{2} t} > 0.
\end{equation}
\end{proof}

\subsection{Quasimodular forms and derivatives}

Morally, quasimodular forms can be thought as \emph{modular forms with differentiations}.
It can be defined formally as follows.

\begin{definition}[Quasimodular form]\label{def:quasimodform}
Let $f: \mathfrak{H} \to \C$ be a holomorphic function, and let $k$ and $s \ge 0$ be integers.
The function $f$ is a \emph{quasimodular form of weight $k$, level $\Gamma$, and depth $s$} if there exist holomorphic functions $f_0, \dots, f_s : \mathfrak{H} \to \C$ such that
\begin{equation}\label{eqn:quasimod-def}
    (f|_{k}\gamma)(z) = (cz + d)^{-k} f\left(\frac{az + b}{cz + d}\right) = \sum_{j=0}^{s} f_j(z) \left(\frac{c}{cz + d}\right)^j
\end{equation}
for all $z \in \mathfrak{H}$ and $\gamma = \left(\begin{smallmatrix} a&b\\c&d \end{smallmatrix}\right) \in \Gamma$.
\end{definition}

By taking $\gamma = \left(\begin{smallmatrix} 1 & 0 \\ 0 & 1 \end{smallmatrix}\right)$, one can check that we should have $f_0 = f$. Thus, a quasimodular form of depth $0$ is just a modular form of same weight and level.


\begin{definition}\label{def:derivative}
Let $F$ be a quasimodular form.
We define the (normalized) derivative of $F$ as
\begin{equation}\label{eqn:derivative}
    F' = DF := \frac{1}{2\pi i} \frac{\dd}{\dd z} F.
\end{equation}
\end{definition}

\begin{lemma}\label{lemma:der-q-series}\uses{def:derivative}
We have an equality of operators $D = q \frac{\dd}{\dd q}$.
In particular, the $q$-series of the derivative of a quasimodular form $F(z) = \sum_{n \ge n_0} a_n q^n$ is $F'(z) = \sum_{n \ge n_0} n a_n q^n$.
\end{lemma}
\begin{proof}
Directly follows from the definition \eqref{def:derivative}, where $\frac{1}{2 \pi i}\frac{\dd}{\dd z}e^{2\pi i n z} = n e^{2\pi i n z}$.
\end{proof}

\begin{theorem}\label{thm:quasimod-der-closed}\uses{def:quasimodform, def:derivative}
The space of quasimodular forms is closed under the derivative \eqref{eqn:derivative}.
If $F$ is a quasimodular form of weight $k$, level $\Gamma$, and depth $s$, then $F'$ is a quasimodular form of weight $k + 2$, level $\Gamma$, and depth $s + 1$.
\end{theorem}
\begin{proof}
This follows from differentiating the definitional equation \eqref{eqn:quasimod-def}: we get
\begin{align}
    &-kc(cz + d)^{-k-1} F\left(\frac{az + b}{cz + d}\right) + (cz + d)^{-k-2} \frac{\dd F}{\dd z} \left(\frac{az + b}{cz + d}\right) \\
    &= \sum_{j=0}^{s} \frac{\dd F_j}{\dd z}(z)\left(\frac{c}{cz + d}\right)^j + F_j(z) \cdot (-j) \left(\frac{c}{cz + d}\right)^{j + 1} \\
    &\Rightarrow (cz + d)^{-k-2}F'\left(\frac{az + b}{cz + d}\right) = \sum_{j=0}^{s + 1} G_j(z) \left(\frac{c}{cz + d}\right)^j
\end{align}
where
\begin{equation}
    G_j(z) := F_j'(z) - \frac{ikc}{2\pi(cz + d)} F_j(z) + \frac{i(j-1)}{2\pi} F_{j-1}(z), \quad F_{-1} := 0
\end{equation}
for $0 \le j \le s + 1$, which are holomorphic.
\end{proof}

The most important quasimodular form is the weight 2 Eisenstein series $E_2$.
We define it as a $q$-series, which gives a holomorphic function on $\mathfrak{H}$.
\begin{definition}\label{def:E2} % \lean{def:E2}
We set
\begin{equation}
    E_2(z):= 1-24\sum_{n=1}^\infty \sigma_1(n)\,e^{2\pi i n z}.\notag
\end{equation}
\end{definition}

\begin{lemma}\label{lemma:E2-transform}\uses{def:E2}
This function is not modular, however it satisfies
\begin{equation}\label{eqn:E2-S-transform}
    z^{-2}\,E_2\Big(\frac{-1}{z}\Big)=E_2(z) -\frac{6i}{\pi}\, \frac{1}{z}.
\end{equation}
More generally, we have
\begin{equation}\label{eqn:E2-transform-general}
(cz + d)^{-2} E_2\left(\frac{az + b}{cx + d}\right) = E_2(z) - \frac{6ic}{\pi (cz + d)}, \quad \begin{pmatrix} a & b \\ c & d\end{pmatrix} \in \mathrm{SL}_{2}(\mathbb{Z}).
\end{equation}
\end{lemma}

\begin{proof}
Use \eqref{eqn:disc-logder}.
Modularity of $\Delta(z)$ gives $(cz + d)^{-12}\Delta(\frac{az + b}{cz + d}) = \Delta(z)$ for $\left(\begin{smallmatrix}a&b\\c&d\end{smallmatrix}\right) \in \Gamma_1$, and by differentiating it we get
\begin{equation}
    (cz + d)^{-14} \Delta'\left(\frac{az + b}{cz + d}\right) = \Delta'(z) - \frac{6ic}{\pi(cz + d)} \Delta(z).
\end{equation}
Now, divide both sides with $\Delta(z)$ proves \eqref{eqn:E2-transform-general}.
\end{proof}

\begin{definition}\label{def:serre-der}\uses{def:derivative, def:E2}
For $k \in \mathbb{R}$, define the weight $k$ Serre derivative $\partial_{k}$ of a modular form $F$ as
\begin{equation}\label{eqn:serre-der}
    \partial_{k}F := F' - \frac{k}{12} E_2 F.
\end{equation}
\end{definition}

\begin{theorem}\label{thm:serre-der-modularity}\uses{def:serre-der, def:holomorphic-modular-form}
Let $F$ be a modular form of weight $k$ and level $\Gamma$.
Then, $\partial_{k}F$ is a modular form of weight $k + 2$ of the same level.
\end{theorem}
\begin{proof}
Let $G = \partial_{k}F = F' - \frac{k}{12}E_2 F$.
It is enough to show that $G$ is invariant under $|_{k+2}\gamma$ for $\gamma \in \Gamma$.
From $F \in M_k(\Gamma)$, we have
\begin{equation}
    (F|_{k}\gamma)(z) := (cz + d)^{-k} F\left(\frac{az + b}{cz + d}\right) = F(z), \quad \gamma = \begin{pmatrix}a & b \\ c & d\end{pmatrix} \in \Gamma.
\end{equation}
By taking the derivative of the above equation, we get
\begin{align}
    &-kc (cz + d)^{-k - 1} F\left(\frac{az + b}{cz + d}\right) + (cz + d)^{-k} (cz + d)^{-2} \frac{\dd F}{\dd z}\left(\frac{az + b}{cz + d}\right) = \frac{\dd F}{\dd z}(z) \\
    &\Leftrightarrow (cz + d)^{-k - 2} F'\left(\frac{az + b}{cz + d}\right) = F'(z) - \frac{ikc}{2\pi(cz + d)}F(z).
\end{align}
Combined with \eqref{eqn:E2-transform-general}, we get
\begin{align}
    ((\partial_k F)|_{k+2}\gamma)(z) &= (cz + d)^{-k-2} \left(F'\left(\frac{az + b}{cz + d}\right) - \frac{k}{12}E_2\left(\frac{az + b}{cz + d}\right)F\left(\frac{az + b}{cz + d}\right)\right) \\
    &= F'(z) - \frac{ikc}{2 \pi(cz + d)} F(z) - \frac{k}{12} \left(E_2 - \frac{6c}{\pi(cz + d)}\right) F(z) \\
    &= F'(z) - \frac{k}{12} E_2(z) F(z) = (\partial_{k} F)(z)
\end{align}
so $\partial_{k}F \in M_{k+2}(\Gamma)$.
\end{proof}
\begin{remark}
% In case we would like to reference this result at any point, I thought we could make it a remark
More generally, the following theorem holds: if $F$ is a quasimodular form of weight $k$ and depth $s$, then $\partial_{k-s}F$ is a quasimodular form of weight $k + 2$ \emph{and depth $\le s$} of the same level. We will not prove this here.
\end{remark}

\begin{theorem}\label{thm:ramanujan-formula}\uses{thm:serre-der-modularity, def:serre-der, lemma:E2-transform}
We have
\begin{align}
    E_2' &= \frac{E_2^2 - E_4}{12} \label{eqn:DE2} \\
    E_4' &= \frac{E_2 E_4 - E_6}{3} \label{eqn:DE4} \\
    E_6' &= \frac{E_2 E_6 - E_4^2}{2} \label{eqn:DE6}
\end{align}
\end{theorem}
\begin{proof}
In terms of Serre derivatives, these are equivalent to
\begin{align}
    \partial_{1}E_2 &= -\frac{1}{12} E_4 \label{eqn:SE2} \\
    \partial_{4}E_4 &= -\frac{1}{3} E_6 \label{eqn:SE4} \\
    \partial_{6}E_6 &= -\frac{1}{2} E_4^2 \label{eqn:SE6}
\end{align}
By Theorem \ref{thm:serre-der-modularity}, all the serre derivatives are, in fact, modular.
To be precise, the modularity of $\partial_{4} E_4$ and $\partial_6 E_6$ directly follows from Theorem \ref{thm:serre-der-modularity}, and that of $\partial_{1}E_2$ follows from \eqref{eqn:E2-transform-general}.
Differentiating and squaring then gives us the following:
\begin{align}
    E_2'|_{4}\gamma &= E_2' - \frac{ic}{\pi(cz + d)} E_2 - \frac{3c^2}{\pi^2 (cz + d)^2} \label{eqn:DE2-transform} \\
    E_2^2|_{4}\gamma &= E_2^2 - \frac{12ic}{\pi(cz + d)} E_2 - \frac{36c^2}{\pi^2 (cz + d)^2} \label{eqn:E2sq-transform}
\end{align}
Hence, \eqref{eqn:DE2}$-\frac{1}{12}$\eqref{eqn:E2sq-transform} is a modular form of weight 4.
Since $\dim M_k(\Gamma_1) = 1$ for $k = 4, 6, 8$, they should be multiples of $E_4, E_6, E_8$, and the proportionality constants can be determined by observing the constant terms of $q$-expansions.
\end{proof}
\begin{corollary}\label{cor:logder-disc-E2}\uses{thm:ramanujan-formula, def:disc-definition}
\begin{equation}\label{eqn:logder-disc-E2}
    \Delta' = E_2 \Delta.
\end{equation}
\end{corollary}
\begin{proof}
By Ramanujan's formula \eqref{eqn:DE4} and \eqref{eqn:DE6},
\begin{equation}
\Delta' = \frac{3 E_4^2 E_4' - 2 E_6 E_6'}{1728} = \frac{1}{1728} \left(3 E_4^2 \cdot \frac{E_2 E_4 - E_6}{3} - 2 E_6 \cdot \frac{E_2 E_6 - E_4^2}{2}\right) = \frac{E_2(E_4^3 - E_6^2)}{1728} = E_2\Delta.
\end{equation}
\end{proof}

Similar argument allow us to compute (Serre) derivatives of $H_2, H_3, H_4$.
\begin{proposition}\label{prop:theta-der}\uses{def:serre-der, lemma:theta-transform-S-T}
We have
\begin{align}
    H_2' &= \frac{1}{6} (H_{2}^{2} + 2 H_{2} H_{4} + E_2 H_2) \label{eqn:H2-der}\\
    H_3' &= \frac{1}{6} (H_{2}^{2} - H_{4}^{2} + E_2 H_3) \label{eqn:H3-der}\\
    H_4' &= -\frac{1}{6} (2H_{2} H_{4} + H_{4}^{2} - E_2 H_4) \label{eqn:H4-der}
\end{align}
or equivalently,
\begin{align}
    \partial_{2} H_{2} &= \frac{1}{6} (H_{2}^{2} + 2 H_{2} H_{4}) \label{eqn:H2-serre-der} \\
    \partial_{2} H_{3} &= \frac{1}{6} (H_{2}^{2} - H_{4}^{2}) \label{eqn:H3-serre-der} \\
    \partial_{2} H_{4} &= -\frac{1}{6} (2H_{2} H_{4} + H_{4}^{2}) \label{eqn:H4-serre-der}
\end{align}
\end{proposition}
\begin{proof}
Equivalences are obvious from the definition of the Serre derivative.
By Theorem \ref{thm:serre-der-modularity}, all the Serre derivatives are modular forms of weight 4 and level $\Gamma(2)$.
We have $\dim M_4(\Gamma(2)) = 3$ with basis $H_2^2, H_2 H_4, H_4^2$, and comparing the first three $q$-coefficients give \eqref{eqn:H2-serre-der}, \eqref{eqn:H3-serre-der}, and \eqref{eqn:H4-serre-der}.
\end{proof}

\begin{theorem}\label{thm:serre-der-prod-rule}\uses{def:serre-der}
The Serre derivative satisfies the following product rule: for any quasimodular forms $F$ and $G$,
\begin{equation}
    \partial_{w_1 + w_2} (FG) = (\partial_{w_1}F)G + F (\partial_{w_2}G).
\end{equation}
\end{theorem}
\begin{proof}
It follows from the definition:
\begin{align}
    \partial_{w_1 + w_2} (FG) &= (FG)' - \frac{w_1 + w_2}{12} E_2 (FG) \\
    &= F'G + FG' - \frac{w_1 + w_2}{12} E_2(FG) \\
    &= \left(F' - \frac{w_1}{12}E_2 F\right)G + F \left(G' - \frac{w_2}{12}E_2 G\right) \\
    &= (\partial_{w_1}F)G + F(\partial_{w_2}G).
\end{align}
\end{proof}

We also have the following useful theorem for proving positivity of quasimodular forms on the imaginary axis, which is \cite[Proposition 3.5, Corollary 3.6]{Lee}.
\begin{theorem}\label{thm:anti-serre-der-pos}\uses{def:serre-der, cor:logder-disc-E2}
Let $F$ be a holomorphic quasimodular cusp form with real Fourier coefficients.
Assume that there exists $k$ such that $(\partial_{k}F)(it) > 0$ for all $t > 0$.
If the first Fourier coefficient of $F$ is positive, then $F(it) > 0$ for all $t > 0$.
\end{theorem}
\begin{proof}
By \eqref{eqn:logder-disc-E2}, we have
\begin{align}
    \frac{\dd}{\dd t} \left( \frac{F(it)}{\Delta(it)^{\frac{k}{12}}}\right)
    &= (-2 \pi) \frac{F'(it) \Delta(it)^{\frac{k}{12}} - F(it) \frac{k}{12} E_{2}(it) \Delta(it)^{\frac{k}{12}}}{\Delta(it)^{\frac{k}{6}}} \\
    &= (-2 \pi) \frac{(\partial_{k} F)(it)}{\Delta(it)^{\frac{k}{12}}}  < 0,
\end{align}
hence
\[
t \mapsto \frac{F(it)}{\Delta(it)^{\frac{k}{12}}}
\]
is monotone decreasing.
Because of the assumption on the positivity of the first nonzero Fourier coefficient of $F$, $F(it) > 0$ for sufficiently large $t$ since
\[
F = \sum_{n \geq n_{0}} a_{n} q^{n} \Rightarrow e^{2 \pi n_{0} t} F(it) = a_{n_{0}} + e^{-2 \pi t}\sum_{n\geq n_{0} + 1} a_{n} e^{-2 \pi (n - n_{0} - 1)t}
\]
and $\lim_{t \to \infty} e^{2 \pi n_{0}t} F(it) = a_{n_0} > 0$, hence the result follows.
\end{proof}


\begin{definition}\label{def:weakly-holomorphic-modular-form}
A \emph{weakly-holomorphic modular form} of integer weight $k$ and congruence subgroup $\Gamma$ is a holomorphic function $f:\h\to\C$ such that:
\begin{enumerate}
  \item $f|_k\gamma=f$ for all $\gamma\in\Gamma$
  \item for each $\alpha\in\Gamma_1\;f|_k\alpha$ has Fourier expansion $f|_k\alpha (z)=\sum_{n=n_0}^\infty c_f(\alpha,\frac{n}{n_\alpha})\,e^{2\pi i \frac{n}{n_\alpha}z}$ for some $n_0\in\Z$ and $n_\alpha\in\N$.
\end{enumerate}
\end{definition}
For an $m$-periodic holomorphic function $f$ and $n\in\frac 1m \Z$, we will denote the $n$-th Fourier coefficient of $f$ by $c_f(n)$, so that
$$f(z)=\sum_{n\in\frac 1m \Z} c_f(n)\,e^{2\pi i n z}.$$
We denote the space of weakly-holomorphic modular forms of weight $k$ and group $\Gamma$ by $M_k^!(\Gamma)$. The spaces $M_k^!(\Gamma)$ are infinite dimensional.
Probably the most famous weakly-holomorphic modular form is the \emph{elliptic j-invariant}
$$j\,:=\,\frac{1728\, E_4^3}{E_4^3-E_6^2}. $$
This function belongs to $M_0^!(\Gamma_1)$ and has the Fourier expansion
$$j(z)\,=\,q^{-1} + 744 + 196884\, q + 21493760\, q^2 + 864299970\, q^3 +
  20245856256\, q^4 + O(q^5) $$
where $q=e^{2\pi i z}$.
% Seewoo: This (lemma below, Hardy-Ramanujan) might be deleted later
Using a simple computer algebra system, such as PARI GP or Mathematica, one can compute first hundred terms of this Fourier expansion in just a few seconds. An important question is to find an  asymptotic formula for $c_j(n)$, the $n$-th Fourier coefficient  of $j$. Using the Hardy-Ramanujan circle method \cite{Rademacher38} or the non-holomorphic Poincare series \cite{Petersson32}, one can show that
\begin{lemma}\label{lemma: j Fourier asymptotic}
% \lean{lemma: j Fourier asymptotic}
\begin{equation}\label{eqn: j Fourier asymptotic}
c_j(n)=\frac{2\pi}{n}\sum_{k=1}^\infty \frac{A_k(n)}{k}\,I_1\left(\frac{4\pi \sqrt{n}}{k}\right)\qquad n\in\Z_{>0}\end{equation}
where
$$A_k(n)= \sum_{\substack{h\;\mathrm{mod}\;k\\(h,k)=1}} e^{\frac{-2\pi i}{k}(nh+h')},\quad hh'\equiv -1(\mbox{mod}\;k),$$
and $I_\alpha(x)$ denotes the modified Bessel function of the first kind defined as in \cite[Section~9.6]{Abramowitz}.
\end{lemma}
A similar convergent asymptotic expansion holds for the Fourier coefficients of any weakly holomorphic modular form \cite{Hejhal}, \cite[Propositions~1.10 and~1.12]{Bruinier}. Such a convergent expansion implies effective estimates for the Fourier coefficients.
